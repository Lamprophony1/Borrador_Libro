\fancyhead{}
\fancyfoot{}
\lhead{Introducción}
\cfoot{\thepage}

\chapter{Introducción}

En América Latina, la vigilancia de la calidad del aire representa una tarea pendiente de vital importancia. La región enfrenta un crecimiento económico y urbano acelerado, lo que conlleva un aumento en la emisión de contaminantes atmosféricos. Sin embargo, la infraestructura para monitorear y gestionar estos contaminantes es a menudo insuficiente. Este déficit es particularmente evidente en Paraguay, donde la expansión industrial y la urbanización continúan sin un seguimiento ambiental adecuado.

La calidad del aire que respiramos está directamente relacionada con nuestra salud y bienestar, así como con la salud de los ecosistemas que nos sostienen. La medición precisa de los contaminantes atmosféricos es fundamental para comprender y mitigar los impactos negativos de la contaminación. El propósito de este proyecto es desarrollar una red de monitoreo que utilice tecnología avanzada para medir concentraciones de contaminantes atmosféricos específicos, tales como el dióxido de nitrógeno (NO2), dióxido de carbono (CO2), ozono (O3) y material particulado (PM). Además, se monitorean variables ambientales como la temperatura y la humedad, que juegan un papel crucial en la interpretación de los datos de calidad del aire. La relevancia de este monitoreo radica en el hecho de que estos contaminantes tienen implicaciones directas en la salud humana. Este sistema no solo tiene como objetivo recopilar datos precisos y en tiempo real, sino que también proporciona una interfaz de visualización de estos datos, permitiendo una interpretación clara y oportuna para el usuario final.



%Este capítulo típicamente realiza la presentación de todo el Trabajo Final de Grado (TFG), excepto por las conclusiones que no deben ser adelantadas aquí. Se considera este capítulo como el inicio de la parte textual del informe del trabajo, toda la redacción preliminar a la introducción corresponde así a la parte pretextual del mismo. Debería incluir, generalmente en este orden \cite{sampieri}.


\section{Motivación}
La inspiración para este proyecto surge de una preocupación profunda por el bienestar de las comunidades, especialmente en la región de Alto Paraná. La motivación principal es doble: por un lado, existe una necesidad urgente de abordar la escasez de datos ambientales en América Latina y, por otro, un deseo de contribuir a la mejora de la salud pública y la preservación del medio ambiente a través de la aplicación de tecnologías de monitoreo avanzadas.

En América Latina, y en Paraguay en particular, el monitoreo de la calidad del aire ha sido históricamente limitado, lo que ha llevado a una falta de conciencia y comprensión sobre sus impactos. Esta carencia de información impide la capacidad de los responsables políticos y la sociedad civil para tomar decisiones informadas que protejan la salud pública y respondan a los desafíos ambientales. La región de Alto Paraná, con su dinámica mezcla de urbanización, industria y áreas naturales, representa un microcosmos de los desafíos ambientales que enfrenta el país.

Este libro tiene como objetivo llenar ese vacío informativo, proporcionando no solo datos valiosos sobre la calidad del aire sino también una metodología replicable para el monitoreo ambiental. La motivación se extiende a empoderar a las comunidades locales con el conocimiento y las herramientas necesarias para abogar por un ambiente más limpio y saludable. Además, se busca inspirar a otros investigadores y profesionales a desarrollar y aplicar soluciones innovadoras en sus propias regiones.

Finalmente, la motivación para este libro es también personal y académica. Representa la culminación de un proyecto de grado que no solo tiene la intención de cumplir con un requisito académico sino también de dejar una huella positiva en la sociedad, contribuyendo a un futuro más sostenible para Paraguay y estableciendo un precedente para la acción ambiental en toda América Latina.

\section{Definición del problema}

El problema central que este trabajo busca abordar es la significativa falta de infraestructura y datos sobre la calidad del aire en Alto Paraná, Paraguay, y, por extensión, en muchas regiones de América Latina. A pesar de la creciente industrialización y urbanización, que conllevan un aumento en la emisión de contaminantes atmosféricos, existe una notable ausencia de sistemas de monitoreo ambiental capaces de proporcionar datos fiables y en tiempo real. Esta carencia de información impide una evaluación precisa de la exposición a la contaminación del aire y sus efectos en la salud pública y el medio ambiente.

La región de Alto Paraná, conocida por su importante contribución al PIB nacional a través de la agricultura, la industria y la energía hidroeléctrica, enfrenta el desafío de equilibrar el desarrollo económico con la sostenibilidad ambiental. La contaminación del aire plantea riesgos para la salud humana, incluyendo enfermedades respiratorias y cardiovasculares, y afecta negativamente a la biodiversidad local.

Además, las condiciones climáticas de Alto Paraná, que incluyen altas temperaturas y niveles variables de humedad, pueden exacerbar la concentración y los efectos de los contaminantes. Sin embargo, la falta de un sistema de monitoreo adecuado ha dejado un vacío en la comprensión y gestión de estos problemas ambientales.

Este trabajo final de grado se propone diseñar e implementar un sistema de monitoreo de calidad del aire que no solo aborde la escasez de datos, sino que también sirva como modelo para la implementación de tecnologías similares en otras regiones con desafíos parecidos. La ausencia de un monitoreo efectivo es un obstáculo significativo para la formulación de políticas de salud y ambientales basadas en evidencia, donde este proyecto busca proporcionar las herramientas necesarias para superar este obstáculo.



Definición del problema: \textit {Necesidad de Red de monitoreo de calidad del aire en Alto Paraná.}

\section{Objetivos}
\subsection{Objetivo general}
Desarrollar prototipo de Red de monitoreo de calidad del aire en Alto Paraná.


\subsection{Objetivos específicos}

\begin{enumerate}
\item Conocer los parámetros permisibles de la calidad del aire.
\item Conocer las herramientas físicas para la medición de partículas.
\item Diseñar dispositivos para el monitoreo de los contaminantes de la calidad del aire, como:
\begin{enumerate}
\item Ozono(O3)
\item Monóxido de Carbono(CO)
\item Dióxido de Nitrógeno(NO2)
\item Material Particulado de 2.5 micras
\item Material Particulado de 10 micras
\end{enumerate}
\item Implementar una red de nodos capaces de recolectar variables de calidad del aire.
\item Generar un algoritmo para la comunicación entre servidor y nodos.
\item Almacenar y gestionar los datos obtenidos de la calidad del aire.
\item Monitorear en tiempo real variables recolectadas por los nodos.
\end{enumerate} \pagebreak

\section{Hipótesis}

Se hipotetiza que la implementación de una Red de Monitoreo de Calidad del Aire en Alto Paraná proporcionará datos significativos sobre los niveles de contaminantes atmosféricos. Estos datos facilitarán la toma de decisiones informadas y mejorarán la gestión de políticas de salud y medio ambiente. Se espera que esta red de monitoreo contribuya a una mejor comprensión de los patrones de contaminación y sus impactos, lo que a su vez permitirá desarrollar estrategias más efectivas para abordar los problemas relacionados con la calidad del aire en la región.


\section{Justificación}

La necesidad de monitorear la calidad del aire en Alto Paraná se justifica por varias razones críticas que tienen implicaciones directas en la salud pública, la sostenibilidad ambiental y el desarrollo económico de la región.

Salud Pública: La exposición a contaminantes atmosféricos está vinculada a una serie de problemas de salud, incluyendo enfermedades respiratorias agudas y crónicas, enfermedades cardiovasculares y un aumento en la mortalidad. La implementación de un sistema de monitoreo permitirá identificar y cuantificar estos contaminantes, contribuyendo significativamente a la prevención de riesgos para la salud de la población.

Sostenibilidad Ambiental: Alto Paraná alberga una biodiversidad rica y ecosistemas sensibles que están en riesgo debido a la contaminación. Un sistema de monitoreo proporcionará datos esenciales para la conservación de estos recursos naturales y para el desarrollo de estrategias de mitigación de impactos ambientales.

Desarrollo Económico: La región de Alto Paraná es un motor económico para Paraguay, y su desarrollo sostenible es clave para el futuro del país. La calidad del aire tiene un impacto directo en la calidad de vida y en la capacidad de la región para atraer inversiones y turismo. Por lo tanto, un sistema de monitoreo es esencial para equilibrar el crecimiento económico con la protección ambiental.

Marco Regulatorio: La falta de datos concretos sobre la calidad del aire ha limitado la capacidad del gobierno para desarrollar y aplicar regulaciones efectivas. Este proyecto proporcionará la información necesaria para formular políticas basadas en evidencia y para cumplir con los estándares internacionales de calidad del aire.

Innovación Tecnológica: La adopción de tecnologías avanzadas para el monitoreo ambiental en Paraguay puede servir como un modelo para otras regiones en América Latina que enfrentan desafíos similares, promoviendo así la innovación y el liderazgo tecnológico en la región.

\section{Delimitación del alcance del trabajo.}

El presente trabajo final de grado se enfoca en el diseño, implementación y evaluación de un sistema de monitoreo de calidad del aire en la región de Alto Paraná. El alcance del proyecto está delimitado por los siguientes parámetros:

Geográfico: El estudio se circunscribe a la región de Alto Paraná, sin incluir comparaciones directas con otras regiones de Paraguay o análisis en profundidad de zonas urbanas y rurales diferenciadas dentro de la región.

Temporal: La recopilación de datos y la evaluación del sistema se llevarán a cabo durante un periodo específico, que será determinado por la duración del proyecto de grado.

Metodológico: El proyecto se limitará a la implementación de tecnología de monitoreo y no abarcará intervenciones para reducir los niveles de contaminación. Además, el análisis se centrará en la viabilidad técnica y la precisión de los datos, sin entrar en el diseño de políticas públicas basadas en los resultados.

Impacto: Mientras que se discutirá el potencial impacto del sistema de monitoreo en la formulación de políticas y la salud pública, el alcance del trabajo no incluirá la implementación real de políticas ni el seguimiento de cambios en la salud pública tras la introducción del sistema.

Esta delimitación asegura que el trabajo se mantenga enfocado y manejable, permitiendo una profundización en los aspectos clave del diseño e implementación del sistema de monitoreo de calidad del aire, y proporcionando una base sólida para futuras investigaciones que puedan expandir el alcance del estudio inicial.



\section{Descripción de los contenidos por capítulo.} 
	\item \textbf{Capítulo 1. Introducción: }Este capítulo establecerá el contexto del proyecto, resaltando la importancia de monitorear la calidad del aire en Alto Paraná. Se expondrán la motivación, la justificación, la hipótesis y los objetivos del estudio, así como la delimitación del alcance del trabajo. Se ofrecerá una visión general del problema de la calidad del aire en la región y cómo este proyecto busca abordarlo.
 
	\item  \textbf{Capítulo 2. Marco Teórico: }En este capítulo se proporcionará una revisión exhaustiva de la literatura sobre la calidad del aire y sus impactos en la salud y el medio ambiente. Se explorarán los conceptos fundamentales de la contaminación atmosférica, los tipos de contaminantes y sus fuentes, así como los estándares internacionales de calidad del aire.
 
	\item  \textbf{Capítulo 3. Herramientas Utilizadas: }En este capítulo se describirán en detalle las herramientas tecnológicas empleadas en el proyecto. Se incluirá una descripción del hardware y del software. Se explicará la selección de estas herramientas, su configuración y su integración en el sistema de monitoreo.\gls{anpr}.
 
	\item  \textbf{Capítulo 4: Resultados: }En este capítulo se presentará los datos recogidos durante el periodo de estudio. Se mostrarán las mediciones de los contaminantes atmosféricos y las condiciones ambientales, y se ilustrará cómo estos datos se visualizan a través de una interfaz. Se incluirán gráficos, tablas y otras representaciones visuales para facilitar la comprensión de los resultados.
 
	\item  \textbf{Capítulo 5: Discusión: }En este capítulo se interpretará los resultados obtenidos, comparándolos con la hipótesis y los objetivos del estudio. Se evaluará la eficacia del sistema de monitoreo y se discutirán las implicaciones de los hallazgos para la salud pública y la política ambiental. Este capítulo también abordará las limitaciones del estudio, sugerencias para investigaciones futuras y recomendaciones basadas en los resultados del proyecto.
 

