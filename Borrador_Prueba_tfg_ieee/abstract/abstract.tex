\thispagestyle{empty}
\begin{center}
\begin{LARGE}
\textbf{Abstract}
\end{LARGE}
\end{center}

\begin{quotation}
To address the growing concern about air quality in Alto Paraná and the lack of a monitoring network, we have developed an environmental monitoring network that measures key pollutants such as nitrogen dioxide (NO2), carbon dioxide ( CO2), ozone (O3) and particulate matter (PM). In addition, temperature and humidity sensors are integrated since they are critical factors that affect the chemistry of some air pollutants.

The heart of our monitoring network is made up of the Heltec ESP32 microcontroller, which has been selected for its robustness, processing capacity with the integrated LoRaWAN module. The latter allows long-range, low-power wireless communication, ideal for deploying a sensor network in large urban areas. The system is powered by a direct current to alternating current converter, guaranteeing continuous and reliable operation.

The proposed system offers a robust solution to address air quality challenges in Alto Paraná. By providing accurate and up-to-date data on air pollutants and environmental conditions, the project supports pollution awareness, regulation and mitigation efforts. Furthermore, it establishes a basis for future research and measures of environmental sustainability in the region.

This work not only details the design and technical implementation of the monitoring system but also highlights the importance of accessibility and understanding of environmental data. The ability to visualize and analyze this data is essential to drive positive changes and protect the health and well-being of local populations, as well as the integrity of the Alto Paraná ecosystem.

\vspace*{0.5cm}

\noindent {\bf Key words:} 1. Monitoring network, 2. Air quality
,3. Air pollutants, 4. Sensors.
\end{quotation}
