\fancyhead{}
\fancyfoot{}
\newtheorem{teorema}{Teorema}
\cfoot{\thepage}

\lhead{Conceptos fundamentales y antecedentes}
%\rhead{\today}
%\rfoot{\thepage}

\chapter{Conceptos fundamentales y antecedentes}
\section{Conceptos fundamentales}

\subsection{Medición}
La medición de la calidad del aire se realiza mediante el uso de sensores de partículas, los cuales son calibrados por el fabricante para garantizar la precisión de los datos. Estos sensores capturan información específica sobre la concentración de contaminantes atmosféricos en el aire. Los datos recogidos por los sensores son leídos y procesados por una microcomputadora, que desempeña un papel crucial en la transmisión de esta información a un servidor con acceso a Internet.

En el servidor, los datos son analizados y procesados para calcular el denominado "índice de calidad del aire" (AQI, por sus siglas en inglés). Este índice permite clasificar la calidad del aire en seis categorías, que van desde "Buena" hasta "Peligrosa". Esta clasificación se basa en el grado de contaminación y sigue las directrices establecidas por la Agencia de Protección Medioambiental de los Estados Unidos (EPA). El AQI es una herramienta valiosa para informar al público sobre la calidad del aire en tiempo real y para tomar decisiones relacionadas con la salud y el medio ambiente.

\subsection{Sensores}

Un sensor es un dispositivo diseñado para detectar y medir una magnitud física o química, conocida como variable de instrumentación, y convertirla en una señal eléctrica que puede ser procesada, almacenada o transmitida según la finalidad definida por el usuario. En el contexto de la medición de la calidad del aire, los sensores juegan un papel crucial al proporcionar datos precisos y en tiempo real sobre diversos contaminantes atmosféricos.

Los sensores pueden funcionar de diversas maneras, dependiendo de su aplicación específica:
\begin{itemize}
\item Algunos sensores proporcionan una lectura directa en la unidad de interés, lo que facilita una interpretación rápida y sencilla de los datos.

\item Otros sensores se conectan a instrumentos que leen la señal y la traducen a la unidad deseada, permitiendo una mayor flexibilidad en la presentación de los datos.

\item También existen sensores que se conectan a dispositivos capaces de almacenar la señal para su procesamiento posterior, lo que es esencial para el análisis a largo plazo y la tendencia de los datos.
\end{itemize}
Los sensores se clasifican según su aplicación, el tipo de señal de salida que generan, o más comúnmente, según la variable física o química que miden. Por ejemplo, en el monitoreo de la calidad del aire, se utilizan sensores específicos para medir concentraciones de contaminantes como el dióxido de nitrógeno (NO2), dióxido de carbono (CO2), ozono (O3) y material particulado (PM).

Con el avance de la tecnología electrónica, los sensores no solo se limitan a traducir cantidades físicas en visualizaciones más simples, sino que también se han integrado en una amplia gama de campos tecnológicos, desde aplicaciones industriales hasta dispositivos de consumo.

\subsection{Red de Sensores}
Una red de sensores es un conjunto de dispositivos distribuidos espacialmente diseñados para medir variables ambientales y físicas, y comunicar esta información a través de nodos interconectados. Estas redes son fundamentales para recopilar datos a gran escala, permitiendo un monitoreo detallado y en tiempo real de condiciones específicas.

\subsection{Ventajas:}

\begin{itemize}
\item Monitoreo en Tiempo Real: Las redes de sensores proporcionan datos actualizados continuamente, esenciales para la detección oportuna de niveles peligrosos de contaminantes y para la respuesta rápida a eventos ambientales.

\item Cobertura Espacial Amplia: Permiten el monitoreo de grandes áreas, superando las limitaciones de las estaciones de monitoreo fijas y aisladas, y proporcionando una visión más completa de la calidad del aire en una región.

\item Resolución de Datos Alta: La capacidad de desplegar múltiples sensores aumenta la resolución de los datos, permitiendo una comprensión más detallada de las variaciones locales en la calidad del aire.

\item Flexibilidad y Escalabilidad: La capacidad de desplegar múltiples sensores aumenta la resolución de los datos, permitiendo una comprensión más detallada de las variaciones locales en la calidad del aire.
\end{itemize}


\subsection{Desventajas:}
\begin{itemize}
\item Mantenimiento y Calibración: : Los sensores requieren mantenimiento regular y calibración para mantener la precisión de los datos, lo que puede ser logísticamente desafiante.

\item Limitaciones de los Sensores de Bajo Costo:  Aunque son más accesibles, pueden ofrecer menor precisión y fiabilidad que los equipos de monitoreo más sofisticados.

\item Vida Útil de la Batería: Los sensores inalámbricos dependen de baterías que necesitan ser reemplazadas o recargadas periódicamente.

\item Interferencias y Pérdida de Datos: Las condiciones ambientales adversas o la interferencia electromagnética pueden afectar la transmisión de datos.

\item Gestión de Datos:  La gran cantidad de datos generados requiere sistemas avanzados para su procesamiento, análisis y almacenamiento.
\end{itemize}

\subsection{Calidad del Aire}
La calidad del aire se refiere a la condición del aire en nuestro entorno y cómo esa condición afecta a la salud de los seres vivos y al medio ambiente. Se mide generalmente en términos de concentraciones de contaminantes específicos, como partículas en suspensión (PM2.5 y PM10), dióxido de nitrógeno (NO2), dióxido de carbono (CO2), ozono (O3), entre otros. La calidad del aire es un indicador de la salud ambiental y se ha convertido en un área de interés público y científico debido a su impacto directo en la salud humana y los ecosistemas.

Historia y Evolución:
El interés por la calidad del aire surgió con mayor fuerza durante la Revolución Industrial, cuando la quema de carbón en las fábricas y la acumulación de smog se convirtieron en problemas evidentes. Sin embargo, no fue sino hasta mediados del siglo XX, después de episodios graves de contaminación como el Gran Smog de Londres en 1952, que se empezaron a establecer las primeras regulaciones significativas. Desde entonces, la calidad del aire ha sido un área de estudio en constante evolución, con avances en la medición, el análisis y la regulación.

\subsection{Ventajas:}

La monitorización de la calidad del aire ofrece múltiples beneficios, incluyendo la protección de la salud pública mediante la identificación y mitigación de riesgos asociados a la contaminación, la conservación de ecosistemas y biodiversidad, el apoyo al desarrollo de políticas y regulaciones ambientales, y la promoción de la conciencia y educación pública sobre temas de salud y medio ambiente.

\subsection{Desventajas:}

Los desafíos asociados con el monitoreo de la calidad del aire incluyen el costo de implementación y mantenimiento, la complejidad técnica en la interpretación de datos, la variabilidad espacial y temporal que requiere monitoreo constante, y la necesidad de acciones complementarias para abordar efectivamente los problemas de contaminación identificados.

\subsection{Internet de las Cosas}
El concepto de IoT, que emergió en la década de 1990, ha ganado prominencia con el avance en la conectividad a Internet, la miniaturización de la electrónica y el desarrollo de estándares de comunicación. En la actualidad, IoT es un componente esencial de la industria 4.0, desempeñando un papel transformador en varios sectores, incluido el monitoreo ambiental.

\subsection{Ventajas:}

\begin{itemize}
\item Automatización y Control Remoto: Permite la gestión automatizada de dispositivos de monitoreo y la respuesta rápida a los cambios en la calidad del aire.

\item Eficiencia y Precisión: Aumenta la eficiencia en la recopilación de datos y mejora la precisión mediante el uso de múltiples sensores.

\item Análisis Avanzado: Facilita el análisis avanzado de grandes volúmenes de datos ambientales a través de plataformas de IoT que integran inteligencia artificial y aprendizaje automático.
\end{itemize}



\subsection{Desventajas:}

\begin{itemize}
\item Interoperabilidad: La diversidad de dispositivos y plataformas puede llevar a problemas de compatibilidad y estandarización.

\item Dependencia de la Conectividad: La efectividad de IoT está ligada a la disponibilidad y fiabilidad de las conexiones a Internet.

\item Costos de Implementación: Aunque los costos están disminuyendo, la implementación de soluciones de IoT puede requerir una inversión inicial significativa.
\end{itemize}


\section{Antecedentes}

\subsection{Diseño e implementación de un sistema de monitoreo de calidad de aire}
Un antecedente relevante en el campo del monitoreo de la calidad del aire en Paraguay es el proyecto liderado por el Ing. Diego Palacios y financiado por el Consejo Nacional de Ciencia y Tecnología (CONACYT) \cite{sistema_de_monitoreo}. Este proyecto se centró en el diseño y desarrollo de un sistema avanzado para monitorear la calidad del aire, empleando tecnologías de información y comunicación para la recopilación y transmisión automática de datos. Se instalaron estaciones de telemetría en ubicaciones estratégicas, incluyendo la Facultad de Ingeniería de la Universidad Nacional de Asunción y el Centro de Innovación Tecnológica en Luque. Estas estaciones midieron en tiempo real los niveles de varios contaminantes atmosféricos y variables meteorológicas, proporcionando datos valiosos sobre la calidad del aire en la región. Este sistema no solo ha contribuido a la comprensión de la calidad del aire en Paraguay, sino que también ha establecido un modelo para futuros proyectos de monitoreo ambiental en América Latina.

\subsection{Diseño e implementación de un sistema IOT para monitorear calidad de aire}
Un proyecto relevante en el ámbito del monitoreo de calidad del aire utilizando tecnologías IoT es el desarrollado en Bogotá, Colombia, como se documenta en el Repositorio Institucional Séneca de la Universidad de los Andes \cite{Sistema_IOT_monitoreo}. Este proyecto, realizado en 2021, implementó una red de monitoreo de calidad del aire que integraba nodos de monitoreo estáticos y móviles, proporcionando mediciones en tiempo real de varios contaminantes atmosféricos. La comunicación entre los nodos se realizó a través de Wi-Fi y MQTT, con una centralización de datos en Google Cloud Platform (GCP), lo que permitió un análisis detallado y en tiempo real de la calidad del aire en la ciudad.

\subsection{Implementación de una Red de Monitoreo de Material Particulado MP2,5 y MP10 en la Ciudad de Asunción}
En el proyecto \cite{Investigacion_niveles_MP} se trabajó con la integración de las distintas plataformas a nivel hardware como software para la implementación de una red de monitoreo de material particulado MP2.5 y MP10 con el objetivo de lograr la adquisición, procesamiento y transmisión automática de datos y parámetros ambientales (temperatura, humedad relativa y presión atmosférica), obteniendo de esta manera información sobre la concentración de MP en 11 sitios específicos en tiempo real, por un período ininterrumpido de 12 meses. Las mediciones realizadas han demostrado que los valores de MP son influenciados no solo por el tráfico vehicular, sino también por fenómenos naturales y otros de origen antropogénico, como la quema de pastizales, una práctica muy común en el país, sobre todo en la época invernal.

\subsection{Monitoreo de bajo costo de parámetros permisibles de la Calidad de Aire}
En el proyecto \cite{Monitoreo_bajo_costo} se desarrolló un prototipo detector de Material Particulado y gases presentes en el ambiente para determinar la calidad del aire, basado en hardware libre y sensores de bajo costo operativo. Se obtuvo la funcionalidad deseada, lo cual indica la posibilidad de aplicación para realizar controles estimativos de la Calidad Del Aire en distintos puntos de la ciudad.
