\thispagestyle{empty}
\begin{center}
\begin{LARGE}
\textbf{Resumen}
\end{LARGE}
\end{center}
\begin{quotation}

Ante la creciente preocupación por la calidad del aire en el Alto Paraná y la ausencia de una red de monitoreo adecuada, se ha desarrollado una red de monitoreo ambiental para medir contaminantes clave como el dióxido de nitrógeno (NO2), dióxido de carbono (CO2), ozono (O3) y material particulado (PM). Se integran también sensores de temperatura y humedad, factores críticos que afectan la química de algunos contaminantes del aire.

El núcleo de esta red de monitoreo consiste en el microcontrolador ESP32 de Heltec, seleccionado por su robustez y capacidad de procesamiento. Este microcontrolador tiene integrado un módulo LoRaWAN que hace la transmisión de datos a través de The Things Network (TTN), aprovechando su eficiencia en el consumo de energía y su capacidad para operar en redes de sensores distribuidos. El sistema se alimenta mediante un convertidor de corriente continua a corriente alterna, asegurando una operación continua y fiable.

El sistema propuesto ofrece una solución robusta para los desafíos de la calidad del aire en Alto Paraná. Proporcionando datos precisos y actualizados sobre los contaminantes atmosféricos y las condiciones ambientales, el proyecto apoya esfuerzos de concienciación, regulación y mitigación de la contaminación. Además, establece una base para futuras investigaciones y medidas de sostenibilidad ambiental en la región.

Este trabajo detalla el diseño y la implementación técnica del sistema de monitoreo y destaca la importancia de la accesibilidad y la comprensión de los datos ambientales. La capacidad de visualizar y analizar estos datos es fundamental para impulsar cambios positivos y proteger la salud y el bienestar de las poblaciones locales, así como la integridad del ecosistema de Alto Paraná.

\vspace*{0.5cm}

\noindent {\bf Descriptores:} 1. Red de Monitoreo, 2. Calidad del Aire, 
3. Contaminantes del aire, 4. Sensores.

\end{quotation}
